These latex files are intended to serve as a the template for the NLP course at FRI.  The template is adapted from the FRI Data Science Project Competition. template  If you find mistakes in the template or have problems using it, please consult Jure Demšar (\href{mailto:jure.demsar@fri.uni-lj.si}{jure.demsar@fri.uni-lj.si}).
In the Introduction section you should write about the relevance of your work (what is the purpose of the project, what will we solve) and about related work (what solutions for the problem already exist). Where appropriate, reference scientific work conducted by other researchers. For example, the work done by Demšar et al. \cite{Demsar2016BalancedMixture} is very important for our project. The abbreviation et al. is for et alia, which in latin means and others, we use this abbreviation when there are more than two authors of the work we are citing. If there are two authors (or if there is a single author) we just write down their surnames. For example, the work done by Demšar and Lebar Bajec \cite{Demsar2017LinguisticEvolution} is also important for successful completion of our project.
