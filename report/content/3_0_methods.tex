PEFT methods can be grouped into five main categories. Additive fine-tuning introduces new trainable parameters for task-specific adaptation, including adapter-based fine-tuning, soft prompt-based fine-tuning, and others. Partial fine-tuning reduces the number of fine-tuned parameters by focusing on critical pre-trained parameters, with methods like bias update, pretrained weight masking, and delta weight masking. Reparameterized fine-tuning utilizes low-rank transformation to decrease trainable parameters, through techniques like low-rank decomposition and LoRA derivatives. Hybrid fine-tuning combines multiple PEFT approaches to leverage strengths and mitigate weaknesses, either manually or automatically. Unified fine-tuning provides streamlined frameworks for incorporating diverse fine-tuning methods into cohesive architectures, emphasizing consistency and efficiency across model adaptation without combining multiple methods.

In this analysis, we will focus on four of these methods:
\paragraph{LoRA}

LoRA \cite{hu2021lora} is a method designed for fine-tuning large models. It operates by fixing the weights of the original model and introducing a trainable low-rank decomposition matrix into the LLM architecture. This modified architecture involves fixing the original weights $W_0$ while introducing additional trainable weights $\Delta W$, which can be decomposed into matrices $BA$, where the rank of both $B$ and $A$ is much smaller than that of $W_0$. Consequently, the resulting weights are formulated as $W0 + BA$, significantly reducing the number of parameters that need to be trained. 

Empirical results indicate that LoRA performs comparably or even better than other methods, while requiring a comparable or lower number of trainable parameters. Notably, LoRA drastically reduces the number of trainable parameters, such as in the case of fine-tuning the GPT-3 model, where the parameter count was reduced by 10000 times, accompanied by a threefold reduction in GPU memory requirement. Additionally, LoRA exhibits several benefits, including the ability to train specialized models without introducing latency, as seen in adapter methods. 

Moreover, it enables the deployment of multiple specialized models simultaneously by reducing memory and computational footprint, achieved through maintaining fixed weights for the base model while having several trained decomposition matrices for each specialized task.

LoRa model is trained with configuration showed in Listing \ref{lst:lora_params}.

\begin{lstlisting}[language=Python, caption={LoRa parameters}, label={lst:lora_params}]
lora_config = LoraConfig(
        r=16,
        lora_alpha=32,
        lora_dropout=0.1,
        bias="lora_only",
        task_type="SEQ_CLS"
)
\end{lstlisting}
\paragraph{Prompt Tuning}
% Prompt tuning\cite{lester2021power} is a process where users refine their input prompts to generate more accurate and relevant responses from AI models like GPT-3.5. It involves experimenting with different phrasings, keywords, and structures to guide the model towards desired outputs. By iteratively adjusting prompts, users can fine-tune the model's understanding and improve the quality of its responses for specific tasks or contexts. Prompt tuning is crucial for optimizing AI performance and tailoring its outputs to meet diverse needs effectively.
summarization of prompt tuning
\paragraph{BitFit} 
Bias-terms Fine-tuning (BitFit) \cite{zaken2022bitfit} is a parameter-efficient fine-tuning technique for pretrained language models that focuses on updating only the bias terms of the model's weights. This approach aims to reduce the computational and memory resources required for fine-tuning while maintaining performance on downstream tasks.

In BitFit, instead of updating all parameters $\mathbf{W}$ and $\mathbf{b}$, we update only the bias $\mathbf{b}$. The weights $\mathbf{W}$ remain fixed. During fine-tuning, the gradients are computed with respect to $\mathbf{b}$ only, and the updates are applied as follows:
\[
\mathbf{b} \leftarrow \mathbf{b} - \eta \frac{\partial \mathcal{L}}{\partial \mathbf{b}}
\]
where $\eta$ is the learning rate and $\mathcal{L}$ is the loss function.

BitFit has three key properties: it can match the results of a fully fine-tuned model, it enables tasks to arrive in a stream without requiring simultaneous access to all datasets, and it fine-tunes only a small portion of the model's parameters. Specifically, BitFit trains less than 0.1\% of the total number of parameters, yet it achieves transfer learning performance comparable to, and sometimes better than, fine-tuning the entire network.

\paragraph{Infused Adapter by Inhibiting and Amplifying Inner Activations (IA3)} The IA3 approach rescales inner activations with learned vectors. These learned vectors are injected in the attention and feedforward modules in a typical transformer-based architecture. These learned vectors are the only trainable parameters during fine-tuning, and thus the original weights remain frozen. Dealing with learned vectors (as opposed to learned low-rank updates to a weight matrix like LoRA) keeps the number of trainable parameters much smaller.

Similar to LoRA, IA3 offers several advantages: it efficiently reduces the number of trainable parameters, with IA3 models typically having only about 0.01\% trainable parameters for base models like T0, compared to over 0.1\% for LoRA. Additionally, IA3 maintains frozen pre-trained weights, allowing for the creation of multiple lightweight and portable models for various tasks. Despite its parameter efficiency, models fine-tuned using IA3 demonstrate performance comparable to fully fine-tuned models, without introducing any inference latency.
