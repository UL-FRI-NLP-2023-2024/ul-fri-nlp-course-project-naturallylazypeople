PEFT methods can be grouped into five main categories. Additive fine-tuning introduces new trainable parameters for task-specific adaptation, including adapter-based fine-tuning, soft prompt-based fine-tuning, and others. Partial fine-tuning reduces the number of fine-tuned parameters by focusing on critical pre-trained parameters, with methods like bias update, pretrained weight masking, and delta weight masking. Reparameterized fine-tuning utilizes low-rank transformation to decrease trainable parameters, through techniques like low-rank decomposition and LoRA derivatives. Hybrid fine-tuning combines multiple PEFT approaches to leverage strengths and mitigate weaknesses, either manually or automatically. Unified fine-tuning provides streamlined frameworks for incorporating diverse fine-tuning methods into cohesive architectures, emphasizing consistency and efficiency across model adaptation without combining multiple methods.

In this analysis, we will focus on three of these methods:

\cite{xu2023parameterefficient} \textcolor{red}{this paper presents existing methods and compares them in experiments. Basically, exactly what we want to do as well. Just more comprehensive}
 General overview:
\begin{itemize}
    \item  comprehensive and systematic review of
    PEFT (parameter-efficient fine-tuning method) methods for PLMs (pre-trained language models)
    \item overall objective: reducing the number of fine-tuning parameters and memory usage while achieving comparable performance to full fine-tuning    
    \item full fine-tuning: e model is trained from scratch for the specific target task (expensive, prone to overfitting)
    \item PEFT: selectively updating or modifying specific parts of the PLMs 
\end{itemize}
Five categories of PEFTs:
\begin{enumerate}
    \item \textbf{additive fine-tuning}: introducing new extra trainable parameters for task-specific fine-tuning
    \begin{enumerate}
        \item \textbf{Adapter-based Fine-tuning}: adapter module is incorporated into the transformer, allowing for fine-tuning without modifying the pretrained parameters 
        % (\textit{e.g. Sequential Adapter, Residual Adapter, Parallel Adapter, AdapterDrop, CoDA, Tiny-Attn Adapter, AdapterFusion , MerA, Hyperformer++, AdapterSoup})
        \item \textbf{\textcolor{red}{Soft Prompt-based Fine-tuning}}: trainable continuous vectors, known as soft prompts, are inserted into the input or hidden state of the model. Unlike manually designed hard prompts, soft prompts are generated by searching for prompts in a discrete token space based on task-specific training data 
        % \textit{(e.g. WARP, Prompt-Tuning, Prefix-tuning, P-tuning, SPOT, ATTEMPT, MPT)}
        \item \textbf{Others} 
    \end{enumerate}
    \item \textbf{Partial Fine-tuning}: aims at reducing the number of fine-tuned parameters by selecting a subset of pre-trained parameters that are critical to downstream tasks while discarding unimportant ones
    \begin{enumerate}
        \item \textbf{Bias Update}: only the bias term in the attention layer, feed-forward layer and layer normalization of the transformer is updated
        \item \textbf{Pretrained Weight Masking}: where the pretrained weights are masked using various pruning criterion
        \item \textbf{Delta Weight Masking}: delta weights are masked via pruning techniques and optimization approximation
    \end{enumerate}
    \item \textbf{Reparameterized Fine-tuning}: utilizing low-rank transformation to reduce the number of trainable parameters while allowing operating with high-dimensional matrices (e.g., pretrained weights)
    \begin{enumerate}
        \item \textbf{Low-rank Decomposition}: s finding a lowerrank matrix that captures the essential information of the original matrix while reducing computational complexity and memory usage by reparameterizing the updated delta weight
        \item \textbf{\textcolor{red}{LoRA Derivatives}}: series of PEFT methods that are improved based on LoRA
    \end{enumerate}
    \item \textbf{Hybrid Fine-Tuning}: aim to combine various PEFT approaches, such as adapter, prefix-tuning, and LoRA, to leverage the strengths of each method and mitigate their weaknesses
    \begin{enumerate}
        \item \textbf{Manual Combination}: mainly involves integrating the structure or features of one PEFT method into another PEFT method to enhance performance while achieving parameter efficiency
        \item \textbf{\textcolor{red}{Automatic Combination}}: explores how to configure PEFT methods like adapters, prefixtuning, BitFit, and LoRA to different layers of the transformers automatically using various structure search and optimization approaches
    \end{enumerate}
    \item \textbf{Unified Fine-tuning}: unified framework for finetuning, which streamlines the incorporation of diverse finetuning methods into a cohesive architecture, ensuring consistency and efficiency across the adaptation and optimization of models. Unlike hybrid fine-tuning methods, unified fine-tuning methods typically utilize a single PEFT method rather than a combination of various PEFT methods.
\end{enumerate}

\noindent\todo{EIleen} \cite{zaken2022bitfit} BitFit Overview:
\begin{itemize}
    \item only the bias-terms of the model (or a subset of them) are being modified
    \item  freezing most of the network and fine-tuning only the bias-terms is surprisingly effective
    \item Three key properties
    \begin{enumerate}
        \item match the results of fully fine-tuned model
        \item enable tasks to arrive in a stream, this way it does not require simultaneous access to all datasets
        \item  fine-tune only a small portion of the model’s parameters
    \end{enumerate}
    \item trains less than 0.1\% of the total number of parameters
    \item Bitfit achieves transfer learning performance which is comparable (and sometimes better!) than fine-tuning of the entire network
\end{itemize}
\paragraph{LoRA}

LoRA \cite{hu2021lora} is a method designed for fine-tuning large models. It operates by fixing the weights of the original model and introducing a trainable low-rank decomposition matrix into the LLM architecture. This modified architecture involves fixing the original weights $W_0$ while introducing additional trainable weights $\Delta W$, which can be decomposed into matrices $BA$, where the rank of both $B$ and $A$ is much smaller than that of $W_0$. Consequently, the resulting weights are formulated as $W0 + BA$, significantly reducing the number of parameters that need to be trained. 

Empirical results indicate that LoRA performs comparably or even better than other methods, while requiring a comparable or lower number of trainable parameters. Notably, LoRA drastically reduces the number of trainable parameters, such as in the case of fine-tuning the GPT-3 model, where the parameter count was reduced by 10000 times, accompanied by a threefold reduction in GPU memory requirement. Additionally, LoRA exhibits several benefits, including the ability to train specialized models without introducing latency, as seen in adapter methods. 

Moreover, it enables the deployment of multiple specialized models simultaneously by reducing memory and computational footprint, achieved through maintaining fixed weights for the base model while having several trained decomposition matrices for each specialized task.

LoRa model is trained with configuration showed in Listing \ref{lst:lora_params}.

\begin{lstlisting}[language=Python, caption={LoRa parameters}, label={lst:lora_params}]
lora_config = LoraConfig(
        r=16,
        lora_alpha=32,
        lora_dropout=0.1,
        bias="lora_only",
        task_type="SEQ_CLS"
)
\end{lstlisting}
\paragraph{Prompt Tuning}
Prompt tuning is an advanced technique in the field of natural language processing (NLP) that involves fine-tuning the prompts given to pre-trained language models to optimize their performance for specific tasks. Unlike traditional model training, which may involve adjusting vast numbers of parameters, prompt tuning focuses on crafting or adjusting the input prompts in a way that elicits more accurate or relevant responses from the model. This approach leverages the existing capabilities of large language models, allowing for efficient task-specific adaptation with minimal computational resources.

Mathematically, consider a pre-trained language model \( f_\theta \) with parameters \( \theta \). Given an input sequence \( x \) and a task-specific prompt \( p \), the output is generated as:

\[ y = f_\theta(p, x) \]

where \( p \) is designed or tuned to optimize the model's performance on a particular task. The goal is to find an optimal prompt \( p^* \) that maximizes the performance metric \( \mathcal{M} \) on a validation set \( D_{\text{val}} \):

\[ p^* = \arg\max_p \mathcal{M}(f_\theta(p, x), y) \quad \text{for} \quad (x, y) \in D_{\text{val}} \]

By carefully designing prompts, practitioners can guide the model to generate desired outputs, improve performance on diverse tasks, and enhance the interpretability and controllability of AI systems.

